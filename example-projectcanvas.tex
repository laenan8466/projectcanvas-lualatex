% File: example-projectcanvas.tex
\documentclass{projectcanvas-lualatex}
\usepackage{lipsum}  % For generating placeholder text
\usepackage[hidelinks]{hyperref}


\begin{document}
\thispagestyle{empty}   % No page numbering

\header{Fancy Project Name}{%
\faGlobe \FAspace fancywebsite.tld \sep%
\faGithub \FAspace \href{https://github.com/org-user/projectname}{org-user/projectname} %
}{John Doe}{%
\faEnvelope \FAspace johndoe@gmail.com%
}

% Flexible column layout (first parameter: left column width, second parameter: right column width)
\begin{flexcolumns}{0.6\linewidth}{0.4\linewidth}

  % Left column
  \begin{topicbox}[Zweck / Herausforderung]{boxcolor}
    \lipsum[1]  % Example content
  \end{topicbox}

  \begin{topicbox}[Stakeholder / Nutzer]{boxcolor}
    \lipsum[2]  % Example content
  \end{topicbox}

  % Right column
  \begin{topicbox}[Lösung]{boxcolor}
    Here could be an image or text content.
  \end{topicbox}

  \begin{topicbox}[Resourcen]{boxcolor}
    \lipsum[3]  % Example content
  \end{topicbox}

\end{flexcolumns}

\end{document}